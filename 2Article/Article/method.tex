\clearpage
\newpage

\section{New methods}

%% Présenter le problème et la factorisation matricielle
%%
%%

In this section we present two new algorithms for estimating individual
admixture coefficients and ancestral genotype frequencies assuming $K$ ancestral
populations. In addition to genotypes, the new algorithms require individual
geographic coordinates of sampled individuals.

\paragraph{$Q$ and $G$-matrices} Consider a genotypic matrix, {\bf Y}, recording
data for $n$ individuals at $L$ polymorphic loci for a $p$-ploid species (common
values for $p$ are $p = 1,2$). For autosomal SNPs in a diploid organism, the
genotype at locus $\ell$ is an integer number, 0, 1 or 2, corresponding to the
number of reference alleles at this locus. In our algorithms, disjunctive forms
are used to encode each genotypic value as the indicator of a heterozygote or a
homozygote locus (Frichot et al. 2014). For a diploid organism each genotypic
value $,0,1,2$ is encoded as $100$, $010$ and $001$. For $p$-ploid organisms,
there are $(p+1)$ possible genotypic values at each locus, and each value
corresponds to a unique disjunctive form. While our focus is on SNPs, the
algorithms presented in this section extend to multi-allelic loci without loss
of generality. Moreover, the method can be easily extended to genotype
likelihoods by using the likelihood to encode each genotypic
value~\citep{Korneliussen2014}.

Our algorithms provide statistical estimates for the matrix ${\bf Q} \in
\mathbb{R}^{K \times n}$ which contains the admixture coefficients, ${\bf
  Q}_{i,k}$, for each sampled individual, $i$, and each ancestral population,
$k$. The algorithms also provide estimates for the matrix ${\bf G} \in
\mathbb{R}^{(p+1)L \times K}$, for which the entries, ${\bf G}_{(p+1)\ell + j,
  k}$, correspond to the frequency of genotype $j$ at locus $\ell$ in population
$k$. Obviously, the $Q$ and $G$-matrices must satisfy the following set of
probabilistic constraints

$$
\quad {\bf Q},{\bf G} \geq 0 \, , \quad \sum_{k=1}^K {\bf Q}_{i,k} = 1 \, ,
\quad \sum_{j=0}^p {\bf G}_{(p+1)\ell + j, k} = 1 \, , \quad j = 0,1,\dots, p,
$$
for all $i, k$ and $\ell$. Using disjunctive forms and the law of total
probability, estimates of {\bf Q} and {\bf G} can be obtained by factorizing the
genotypic matrix as follows ${\bf Y}$=${\bf Q}\,{\bf G}^T$~\citep{Frichot2014}.
Thus the inference problem can be solved by using constrained nonnegative matrix
factorization methods~\citep{Lee1999, Cichocki2009}. In the sequel, we shall use
the notations $\Delta_Q$ and $\Delta_G$ to represent the sets of probabilistic
constraints put on the {\bf Q} and {\bf G} matrices respectively.


\paragraph{Geographic weighting} Geography is introduced in the matrix
factorization problem by using weights for each pair of sampled individuals. The
weights impose regularity constraints on ancestry estimates over geographic
space. The definition of geographic weights is based on the spatial coordinates
of the sampling sites, $(x_i)$. Samples close to each other are given more
weight than samples that are far apart. The computation of the weights starts
with building a complete graph from the sampling sites. Then the weight matrix
is defined as follows

$$
w_{ij} = \exp( - {\rm dist}( x_i, x_j )^2/ \sigma^2),
$$
\noindent where dist$( x_i, x_j )$ denotes the geodesic distance between sites
$x_i$ and $x_j$, and $\sigma$ is a range parameter.


Next, we introduce the {\it Laplacian matrix} associated with the geographic
weight matrix, {\bf W}. The Laplacian matrix is defined as ${\bf \Lambda}$ =
{\bf D} $-$ {\bf W} where {\bf D} is a diagonal matrix with entries ${\bf
  D}_{i,i} = \sum_{j = 1}^n {\bf W}_{i,j}$, for $i = 1, \dots,
n$~\citep{Belkin2003}. Elementary matrix algebra shows that~\citep{DengCai2011}

$$
 {\rm Tr} ({\bf Q}^T {\bf \Lambda} {\bf Q})  = \frac12 \sum_{i,j = 1}^n  w_{ij}  \|   {\bf Q}_{i,.}  - {\bf Q}_{j,.} \|^2 \, .
$$
In our approach, assuming that geographically close individuals are more likely to share ancestry than individuals at distant sites is thus equivalent to minimizing the quadratic form ${\cal C}({\bf Q}) ={\rm Tr} ({\bf Q}^T {\bf \Lambda} {\bf Q})$ while estimating the matrix ${\bf Q}$. 

\paragraph{Least-squares optimization problems} Estimating the matrices ${\bf
  Q}$ and ${\bf G }$ from the observed genotypic matrix ${\bf Y}$ is performed
through solving an optimization problem defined as follows~\citep{Caye2016}

\begin{equation}
\begin{aligned}
& \underset{Q, G}{\text{min}}
& & {\rm LS}({\bf Q}, {\bf G}) =   \|  {\bf Y} - {\bf QG}^T \|^2_{\rm F} +  \alpha {\cal C}({\bf Q}) , \\
& \text{s.t.} & &  {\bf Q} \in \Delta_Q , \\
& & &  {\bf G} \in \Delta_G . \\
\end{aligned}
\label{eq:LS}
\end{equation}
\noindent The notation $\| {\bf M} \|_{\rm F}$ denotes the Frobenius norm of a
matrix, {\bf M}. The regularization parameter $\alpha $ controls the regularity
of ancestry estimates over geographic space. Large values of $\alpha $ imply
that ancestry coefficients have similar values for nearby individuals, whereas
small values ignore spatial autocorrelation in observed allele frequencies.

\paragraph{The Alternating Quadratic Programming (AQP) method} Because the
poly\-edrons $\Delta_Q$ and $\Delta_G$ are convex sets and the LS function is
convex with respect to each variable ${\bf Q}$ or ${\bf G}$ when the other one
is fixed, the problem~\eqref{eq:LS} is amenable to the application of block
coordinate descent~\citep{Bertsekas1995}. The APQ algorithm starts from initial
values for the $G$ and $Q$-matrices, and alternates two steps. The first step
computes the matrix {\bf G} while {\bf Q} is kept fixed, and the second step
permutates the roles of {\bf G} and {\bf Q}. Let us assume that {\bf Q} is fixed
and write {\bf G} in a vectorial form, $g = {\rm vec({\bf G})} \in
\mathbb{R}^{K(p + 1)L}$. The first step of the algorithm actually solves the
quadratic programming subproblem, 

\begin{equation}
\begin{aligned}
g^\star = \underset{g \in \Delta_G}{\arg \min}  ( -2  v^T_Q \, g + g^T {\bf D}_Q g ) \, ,  
\end{aligned}
\label{eq:AQPg}
\end{equation}
\noindent where ${\bf D}_Q = {\bf I}_{(p+1)L} \otimes {\bf Q}^T {\bf Q}$ and
$v_Q = {\rm vec}({\bf Q}^T {\bf Y})$. Here, $\otimes$ denotes the Kronecker
product and ${\bf I}_d$ is the identity matrix with $d$ dimensions. The block
structure of the matrix ${\bf D}_Q$ allows us to decompose the
subproblem~\eqref{eq:AQPg} into $L$ independent quadratic programming problems
with $K(p + 1)$ variables. Now, consider that {\bf G} is the value obtained
after the first step of the algorithm, and write {\bf Q} in a vectorial form, $q
= {\rm vec({\bf Q})} \in \mathbb{R}^{nK}$. The second step solves the following
quadratic programming subproblem. Find

\begin{equation}
\begin{aligned}
q^\star = \underset{q \in \Delta_Q}{\arg \min} ( -2 v^T_G \, q + q^T {\bf D}_G q ) \,  ,
\end{aligned}
\label{eq:AQPq}
\end{equation}
\noindent where ${\bf D}_G = {\bf I}_{n} \otimes {\bf G}^T {\bf G } + \alpha
{\bf \Lambda} \otimes {\bf I}_K$ and $v_G = {\rm vec}({\bf G}^T{\bf Y}^T)$.
Unlike subproblem~\eqref{eq:AQPg}, subproblem~\eqref{eq:AQPq} can not be
decomposed into smaller problems. Thus, the computation of the second step of
the AQP algorithm implies to solve a quadratic programming problem with $nK$
variables which can be problematic for large samples ($n$ is the sample size).
The AQP algorithm is described in details in Appendix~\ref{algo:aqp}. For AQP,
we have the following convergence result.
\begin{thm}
\label{th}
	The AQP algorithm converges to a critical point of problem~\eqref{eq:LS}.
\end{thm}
\begin{proof}
  The quadratic convex functions defined in subproblems~\eqref{eq:AQPg}
  and~\eqref{eq:AQPq} have finite lower bounds. The convex sets $\Delta_Q$ and
  $\Delta_G$ are compact non-empty sets. Thus the sequence generated by the AQP
  algorithm is well-defined, and has limit points. According to Corollary 2 of
  ~\cite{Grippo2000}, we conclude that the AQP algorithm converges to a critical
  point of problem~\eqref{eq:LS}.
\end{proof}

\paragraph{Alternating Projected Least-Squares (APLS)} In this paragraph, we
introduce an APLS estimation algorithm which approximates the solution of
problem~\eqref{eq:LS}, and reduces the complexity of the AQP algorithm. The APLS
algorithm starts from initial values of the $G$ and $Q$-matrices, and alternates
two steps. The matrix {\bf G} is computed while {\bf Q} is kept fixed, and {\it
  vice versa}. Assume that the matrix {\bf Q} is known. The first step of the
APLS algorithm solves the following optimization problem. Find

\begin{equation}
{\bf G}^\star = \arg \min  \|  {\bf Y} - {\bf QG}^T \|^2_{\rm F} \, .
\end{equation}
This operation can be done by considering $(p+1)L$ (the number of columns of
${\bf Y}$) independent optimization problems running in parallel. The operation
is followed by a projection of ${\bf G}^\star$ on the polyedron of constraints,
$\Delta_G$. For the second step, assume that {\bf G} is set to the value
obtained after the first step is completed. We compute the eigenvectors, {\bf
  U}, of the Laplacian matrix, and we define the diagonal matrix ${\bf \Delta}$
formed by the eigenvalues of ${\bf \Lambda}$ (The eigenvalues of ${\bf \Lambda}$
are non-negative real numbers). According to the spectral theorem, we have

$$
{\bf \Lambda} = {\bf U}^T {\bf \Delta} {\bf U} \, .
$$
\noindent  After this operation, we project the data matrix {\bf Y} on the basis of eigenvectors as follows

$$
{\rm proj} ({\bf Y}) = {\bf U}{\bf Y} \, , 
$$
\noindent and, for each individual, we solve the following optimization problem

\begin{equation}
q_i^\star = \arg \min  \|  {\rm proj} ({\bf Y})_i  - {\bf G} q \|^2 + \alpha \lambda_i \| q \|^2  \, ,
\label{eq:APSLq}
\end{equation}
\noindent where proj({\bf Y}$)_i$ is the $i$th row of the projected data matrix,
proj({\bf Y}), and $\lambda_i$ is the $i$th eigenvalue of ${\bf \Lambda}$. The
solutions, $q_i$, are then concatenated into a matrix, ${\rm conc}(q)$, and
${\bf Q}$ is defined as the projection of the matrix ${\bf U}^T {\rm conc}(q)$
on the polyedron $\Delta_Q$. The complexity of step~\eqref{eq:APSLq} grows
linearly with $n$, the number of individuals. While the theoretical convergence
properties of AQP algorithms are lost for APLS algorithms, the APLS algorithms
are expected to be good approximations of AQP algorithms. The APLS algorithm is
described in details in Appendix~\ref{algo:apls}.


\paragraph{Choice of hyper-parameters.} Determination of the number of factors,
$K$, the penalty constant, $\alpha$, and the range parameter $\sigma$ is
notoriously difficult. For ancestry estimation, a number of common practices
have evolved to help reducing computational burden. Those practices rely on
heuristics or empirical rules for determining the prior parameters. For example,
the program structure implements weakly informative prior distributions for
ancestry proportions~\citep{wang2017}, the program admixture has a set of
regularization parameters that encourages shrinkage and “aggressive” parsimony
on ancestry estimates~\citep{Alexander2011}, and so does the Bayesian
version TESS 2.3~\citep{Durand2009}. In general, more intense efforts are
devoted to choosing the number of factors, and cross-validation methods or and
information theoretic measures are used to this purpose. In our program
implementation, the hyper-parameters $\alpha$ and $\sigma$ are set as options
with default values, which allows users to explore different values consistent
with cross-validation or known heuristics.

\paragraph{Range parameter.} Testing correlations between genetic and geographic
has a long tradition in population genetics. Popular approaches are based on
Mantel tests~\citep{mantel1967} and spatial autocorrelation
measures~\citep{hardy1999, epperson1996}. Prior to the application of our spatial ancestry
estimation program, we investigated biologically relevant values for the range
parameter by using spatial variograms~\citep{Cressie1993}. The variogram was
extended to genotypic data as follows


\begin{equation}
\gamma(h) = \frac{1}{2 |N(h)|} \sum_{i,j \in N(h)} \frac{1}{L} \sum_{l = 1}^{(p+1)L} |Y_{i,l} - Y_{j,l}|,
\label{eq:gamma}
\end{equation}
where $N(h)$ is defined as the set of individuals separated by geographic
distance $h$. Visualizing the variogram provides useful information on the level
of spatial autocorrelation in the data, and yields empirical estimates of the
range parameter. More naïve estimates such as an average geodesic distance
computed over a fraction of neighboring sites in the sample also performed well
in simulations, and they are also proposed to the program users.

\paragraph{Regularization parameter.} A default value for the regularization
parameter $\alpha$ was set so that the weights for the loss function and for
the penalty term ${\cal C}({\bf Q})$ are of similar order. We proposed to divide
each term by its maximum value. This amount to consider $\alpha$ equal to $L /
\lambda_\max$, where $\lambda_\max$ is the largest eigenvalue of the Laplacian
matrix (The Laplacian matrix has nonnegative eigenvalues).

\paragraph{Number of factors.} The number of ancestral populations, $K$, can be
evaluated by using a cross-validation technique based on imputation of masked
genotypes~\citep{wold1978,eastment1982, Alexander2011, Frichot2014}. The
cross-validation procedure partitions the genotypic matrix entries into a
learning set and a test set in which 5\% of all genotypes are tagged as masked
entries. The genotype probabilities for the masked entries are predicted from
the factor estimates obtained from unmasked entries. Then, a cross-entropy
between the predicted and truly observed genotype frequencies is computed, and
smaller values of that criterion indicate better choices.

\paragraph{Comparison with {\tt tess3}} The algorithm implemented in a previous
version of {\tt tess3} also provides another approximation of the solution
of problem~\eqref{eq:LS}. The {\tt tess3} algorithm first computes a Cholesky
decomposition of the Laplacian matrix. Then, by a change of variables, the
least-squares problem is transformed into a sparse nonnegative matrix
factorization problem~\citep{Caye2016}. Solving the sparse non-negative matrix
factorization problem relies on the application of existing
methods~\citep{Kim2011, Frichot2014}. The methods implemented in {\tt tess3}
have an algorithmic complexity that increases linearly with the number of loci
and the number of clusters. They lead to estimates that accurately reproduce
those of the Monte Carlo algorithms implemented in the Bayesian method {\tt
  tess} 2.3~\citep{Caye2016}. Like for the AQP method, the {\tt tess3} previous
algorithms have an algorithmic complexity that increases quadratically with the
sample size.




\paragraph{Ancestral population differentiation statistics and local adaptation
  scans} Assuming $K$ ancestral populations, the $Q$ and $G$-matrices obtained
from the AQP and from the APLS algorithms were used to compute single-locus
estimates of a population differentiation statistic similar to $F_{\rm
  ST}$, as follows

$$
F^{Q}_{\rm ST} = 1 - \sum_{k=1}^K  q_k \frac{f_k (1-f_k)}{f(1-f)} \, ,
$$
\noindent where $q_k$ is the average of ancestry coefficients over sampled
individuals, $q_k = \sum_{i =1}^n Q_{i,k}/n$, for the cluster $k$, $f_k$ is the
ancestral allele frequency in population $k$ at the locus of interest,

$$
f_k =  \sum_{j = 1}^p  j G_{(p+1)(\ell) + j, k}/p ,
$$
and $f = \sum_{k = 1}^K q_k f_k$~\citep{Martins2016}.
For a particular locus, the formula for $F^Q_{\rm ST}$ corresponds to the
proportion of the genetic variation (or variance) in ancestral allele frequency
that can be explained by latent population structure

$$
F^Q _{\rm ST}  =  \frac{\sigma^2_T - \sigma^2_S}{\sigma^2_T },
$$
where $\sigma^2_T$ is the total variance and $\sigma^2_S$ is the error
variance~\citep{weir1996}. Following ANOVA theory, the $F^Q_{\rm ST}$ statistics
were used to perform statistical tests of neutrality at each locus, by comparing
the observed values to their expectations from the genome-wide background. The
test was based on the squared $z$-score statistic, $z^2 = (n-K) F^{Q}_{\rm
  ST}/(1 - F^{Q}_{\rm ST})$, for which a chi-squared distribution with $K-1$
degrees of freedom was assumed under the null-hypothesis. To avoid an increased
number of false positive tests, we adopted an empirical null-hypothesis testing
approach that recalibrates the null-hypothesis for the background levels of
population differentiation expected at selectively neutral SNPs~\citep{efron2004}.
The calibration of the null-hypothesis was achieved by using genomic control to
adjust the test statistics~\citep{Devlin1999, Francois2016}. After recalibration
of the null-hypothesis, the control of the false discovery rate was achieved by
using the Benjamini-Hochberg algorithm~\citep{Benjamini1995}.


\paragraph{{\tt R} package} We implemented the AQP and APLS algorithms and
improved graphical tools in the {\tt R} package {\tt tess3r}, available from
Github and submitted to the Comprehensive R Archive Network (R Core Team, 2016).


\section{Simulated and real data sets}
\paragraph{Coalescent simulations} We used the computer program {\tt ms} to
perform coalescent simulations of neutral and outlier SNPs under spatial models
of admixture~\citep{Hudson2002}. Two ancestral populations were created from the
simulation of Wright\rq{}s two-island models. The simulated data sets contained
admixed genotypes for $n$ individuals for which the admixture proportions varied
continuously along a longitudinal gradient~\citep{Durand2009, Francois2010}. In
those scenarios, individuals at each extreme of the geographic range were
representative of their population of origin, while individuals at the center of
the range shared intermediate levels of ancestry in the two ancestral
populations~\citep{Caye2016}. For those simulations, the $Q$ matrix, ${\bf
  Q}_0$, was entirely described by the location of the sampled individuals.


Neutrally evolving ancestral chromosomal segments were generated by simulating
DNA sequences with an effective population size $N_0 = 10^6$ for each ancestral
population. The mutation rate per bp and generation was set to $\mu = 0.25
\times 10^{-7}$, the recombination rate per generation was set to $r = 0.25
\times 10^{-8}$, and the parameter $m$ was set to obtained neutral levels of
$F_{\rm ST}$ ranging between values of $0.005$ and $0.10$. The number of base
pairs for each DNA sequence was varied between 10k to 300k to obtain numbers of
polymorphic loci ranging between 1k and 200k after filtering out SNPs with
minor allele frequency lower than 5$\%$. To create SNPs with values in the tail
of the empirical distribution of $F_{\rm ST}$, additional ancestral chromosomal
segments were generated by simulating DNA sequences with a migration rate $m_s$
lower than $m$. The simulations reproduced the reduced levels of diversity and
the increased levels of differentiation expected under hard selective sweeps
occurring at one particular chromosomal segment in ancestral
populations~\citep{Martins2016}. For each simulation, the sample size was varied
in the range $n =$ 50-700.

We compared the AQP and APLS algorithm estimates with those obtained with the
{\tt tess3} algorithm. Each program was run 5 times on the same simulated data.
Using $K = 2$ ancestral populations, we computed the root mean squared error
(RMSE) between the estimated and known values of the $Q$-matrix, and between the
estimated and known values of the $G$-matrix. To evaluate the benefit of spatial
algorithms, we compared the statistical errors of APLS algorithms to the errors
obtained with {\tt snmf} method that reproduces the outputs of the {\tt
  structure} program accurately~\citep{Frichot2014,Frichot2015}. To quantify the
performances of neutrality tests as a function of ancestral and observed levels
of $F_{\rm ST}$, we used the area under the precision-recall curve (AUC) for
several values of the selection rate. Subsamples from a real data set were used
to perform a runtime analysis of the AQP and APLS algorithms ({\it A. thaliana}
data, see below). Runtimes were evaluated by using a single computer processor
unit Intel Xeon 2.0 GHz.

\paragraph{Application to human data.} To evaluate the robustness of our
approach to a situation where admixture was the consequence of large
displacement rather than contact between proximal populations, we studied the
case of African-American populations. This is an interesting case for which the
incorporation of geographic data could potentially bias estimation of ancestry
coefficients. Genotypes with minor allele frequency greater than $5 \%$ were
obtained from a public release of the
% TODO number of snips + number of infiv by pop
1000 Genome project phase 3 (2015) for individuals from African Americans (ASW,
X individuals), Africans (YRI and LWK, X individuals), and Europeans (GBR and
TSI, X individuals)~\citep{1000genome}. A total of 500k SNPs were analyzed with geographic data
corresponding to the country of origin of individual samples. We compared the
estimates from the APLS algorithm applied with its default parameter settings to
the results of the sNMF program that do not make use of geographic information.

\paragraph{Application to European ecotypes of {\it Arabidopsis thaliana}} We
used the APLS algorithm to survey spatial population genetic structure and to
investigate the molecular basis of adaptation by considering 214k SNPs from
1,095 European ecotypes of the plant species {\it A. thaliana}
~\citep{Horton2012}. The cross-validation criterion was used to evaluate the
number of clusters in the sample, and a statistical analysis was performed to
evaluate the range of the variogram from the data. We used {\tt R} functions of
the {\tt tess3r} package to display interpolated admixture coefficients on a
geographic map of Europe (R Core team 2016). A gene ontology enrichment analysis
using the software AMIGO~\citep{Carbon2009} was performed in order to evaluate
which molecular functions and biological processes might be involved in local
adaptation of {\it A. thaliana} in Europe.


