\section{Introduction}

 High-throughput sequencing technologies have enabled studies of genetic ancestry for model and non-model species at an unprecedented pace. In this  context, ancestry estimation algorithms are important for demographic analysis, medical genetics, conservation and landscape genetics~\citep{Pritchard2000, Tang2005, Schraiber2015, Segelbacher2010, Francois2015}.  With increasingly large data sets, Bayesian approaches to the inference of population structure, exemplified by the computer program {\tt structure} \citep{Pritchard2000}, have been replaced by approximate algorithms that run several orders faster than the original version~\citep{Tang2005, Alexander2011, Frichot2014, Raj2014}.  Considering $K$ ancestral populations or genetic clusters, those algorithms estimate ancestry coefficients following two main directions: model-based and model-free approaches. In model-based approaches, a likelihood function is defined for the matrix of ancestry coefficients, and estimation is performed by maximizing the logarithm of the likelihood function. For {\tt structure} and derived models, model assumptions include linkage equilibrium and Hardy-Weinberg equilibrium in ancestral populations. The first approximation to the original algorithm was based on an expectation-minimization algorithm~\citep{Tang2005}, and more recent likelihood algorithms are implemented in the programs {\tt admixture}  and {\tt faststructure} \citep{Alexander2011, Raj2014}. In model-free approaches, ancestry coefficients are estimated by using least-squares methods or factor analysis. Model-free methods make no assumptions about the biological processes that have generated the data. To estimate ancestry matrices, \cite{Engelhardt2010} proposed to use sparse factor analysis, \cite{Frichot2014} used sparse non-negative matrix factorization algorithms, and \cite{Popescu2014} used kernel-principal component analysis. Least-squares methods accurately reproduce the results of likelihood approaches under the model assumptions of those methods~\citep{Frichot2014, Popescu2014}.  In addition, model-free methods provide approaches that are valid when the assumptions of likelihood approaches are not met. Model-free methods are generally faster than model-based methods. 
   
  Among model-based approaches to ancestry estimation, an important class of methods have improved the Bayesian model of {\tt structure} by incorporating geographic data through spatially informative prior distributions~\citep{Chen2007, Corander2008}. Under isolation-by-distance patterns~\citep{Wright1943, Malecot1948}, spatial algorithms provide more robust estimates of population structure than non-spatial algorithms which can lead to biased estimates of the number of clusters~\citep{Durand2009}.  Some Bayesian methods are based on Markov chain Monte Carlo algorithms which are computer-intensive~\citep{Francois2010}. Recent efforts to improve the inference of ancestral relationships in a geographical context have mainly focused on the localization of recent ancestors~\citep{Baran2013, Lao2014, Yang2014}. In these applications, spatial information is used in a predictive framework that assigns ancestors to putative geographic origins. While fast geographic estimation of individual ancestry proportions has been proposed previously~\citep{Caye2016}, there is a growing need to develop individual ancestry estimation algorithms that reduce computational cost in a geographically explicit framework. 

In this study, we present two new algorithms for the estimation of ancestry matrices based on geographic and genetic data. The new algorithms solve a least squares optimization problem as defined by~\cite{Caye2016}, based on Alternating Quadratic Programming  (AQP) and Alternating Projected Least Squares (APLS). While AQP algorithms have a well-established theoretical background~\citep{Bertsekas1995}, this is not the case of APLS algorithms. Using coalescent simulations, we provide evidence that the estimates computed by APLS algorithms are good approximations to the solutions of AQP algorithms. In addition, we show that the performances of APLS algorithms scale with the dimensions of modern data sets. We discuss the application of our algorithms to data from European ecotypes of {\it Arabidopsis thaliana}, for which individual genomic an geographic data are available~\citep{Horton2012}. 



